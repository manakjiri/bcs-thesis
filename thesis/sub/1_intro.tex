%!TEX ROOT=main.tex

With the upward trend in global temperatures, the world is witnessing an increased scarcity of water resources. This environmental shift poses a significant challenge to traditional farming practices, which are heavily reliant on predictable weather patterns and stable water supply. 

In the face of these challenges, the adoption of smart agriculture technologies, emerges as a vital strategy. These technologies offer the potential to transform agricultural practices by optimizing water usage, improving crop yields, and ensuring sustainable farming operations.

This project aims to develop a sensor network that is not only reliable and easy to maintain but also versatile enough to be applied across various contexts - from small-scale home gardens to medium-sized agricultural fields and regional deployments. 

The technical challenges will be the primary focus of this work, and since many of the requirements for a reliable remote sensing solution are common amongst other applications, it will also venture into those, especially in the initial architectural and design stages. 

\section{Goals}
Many options exist when it comes to transferring sensor data over long distances. For this work, the LoRa technology was selected for its unique blend of long-range (orders of 1 to 10 km in favorable conditions), albeit relatively slow (0.3 to 50 kbit \cite{semtech_corporation_sx12612_2024}), data transfer, with very low power consumption (5 mA receiving, even lower in carrier sense mode \cite{semtech_corporation_sx12612_2024}), global availability and general fit for the purpose.

Furthermore, an Over The Air (OTA) firmware update capability will be explored, a feature not commonly found in LoRa nodes. In the case of soil moisture sensors, an update can bring support for additional soil types, improve the accuracy or prolong the battery life. It also, crucially, allows this to happen without any operator intervention and transforms a one-shot solution, which once obsolete is replaced completely, into one that is, over time, able to support more sensors and features with minimal downtime \cite{mobility_connected_what_2024,noauthor_android_2024,bucklin_brown_over--air_2024}.

A custom hardware solution will need to be developed in order to support OTA updates. Chapter \ref{chapter:architecture} in particular will make a point as to why this is necessary by compiling the hardware requirements and assessing the existing solutions currently available on the market. 

This hardware solution will consist of two pieces - a general purpose LoRa module (the ``LoRa module'' from now on) capable of OTA updates and a soil moisture sensor, which integrates the LoRa module into the specialized application.

To implement the sensing, its configuration, status reporting and the update capability, a custom protocol will be developed atop the LoRa physical layer. This will make the system self-sufficient, but may limit the deployment options. However, the system could also be made compatible with majority of commercial LoRaWAN networks in the future, if needed.

Lastly, on the note of autonomy and minimal maintenance, individual nodes should be power-efficient enough to be powered by a small built-in battery augmented by a solar panel, to achieve a ``set and forget'' deployment.

The system will be evaluated based on its performance in the field...
