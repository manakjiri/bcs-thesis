% arara: pdflatex: { synctex: yes }
% arara: makeindex: { style: ctuthesis }
% arara: bibtex

% The class takes all the key=value arguments that \ctusetup does,
% and a couple more: draft and oneside
\documentclass[twoside]{ctuthesis}
\usepackage[toc]{appendix}
\usepackage{graphicx}
\usepackage{subfig}
\usepackage{svg}
\usepackage{placeins}
\usepackage[capposition=bottom]{floatrow}
\usepackage{listings, listings-rust}
\usepackage{multirow}
\usepackage{adjustbox}

\PassOptionsToPackage{hyphens}{url}
\usepackage[pdftex,
            pdfauthor={Jiří Maňák},
            pdftitle={Sensor network for Smart Agriculture}
			]{hyperref}

\ctusetup{
	preprint = \today,
	mainlanguage = english,
	titlelanguage = english,
	otherlanguages = {czech},
	title-czech = {Senzorová síť pro Smart Agriculture},
	title-english = {Sensor network for Smart Agriculture},
%	subtitle-czech = {},
%	subtitle-english = {},
	doctype = B,
	faculty = F3,
	department-czech = {Katedra meření},
	department-english = {Department of Measurement},
	author = {Jiří Maňák},
	supervisor = {prof. Ing. Radislav Šmíd, Ph.D.},
	supervisor-address = {Praha, Technická 1902/2, místnost: A3-324},
%	supervisor-specialist = {},
%	fieldofstudy-english = {},
%	subfieldofstudy-english = {},
%	fieldofstudy-czech = {},
%	subfieldofstudy-czech = {},
	keywords-czech = {slovo, klíč},
	keywords-english = {word, key},
	day = 14,
	month = 5,
	year = 2024,
	specification-file = {sub/thesis-specification.pdf},
%	front-specification = true,
%	front-list-of-figures = false,
%	front-list-of-tables = false,
%	monochrome = true,
%	layout-short = true,
}


\ctuprocess

\addto\ctucaptionsczech{%
\def\supervisorname{prof. Ing. Radislav Šmíd, Ph.D.}%
\def\subfieldofstudyname{Kybernetika a Robotika}%
}

\ctutemplateset{maketitle twocolumn default}{
	\begin{twocolumnfrontmatterpage}
		\ctutemplate{twocolumn.thanks}
		\ctutemplate{twocolumn.declaration}
		\ctutemplate{twocolumn.abstract.in.titlelanguage}
		\ctutemplate{twocolumn.abstract.in.secondlanguage}
		\ctutemplate{twocolumn.tableofcontents}
		\ctutemplate{twocolumn.listoffigures}
	\end{twocolumnfrontmatterpage}
}

\lstset{language=Rust, style=boxed}

\newcommand{\code}[1]{\lstinline{#1}}
\newcommand{\link}[2]{\href{#1}{#2}\footnote{#1}}

\setlength{\parskip}{1.5ex plus 0.2ex minus 0.1ex}

% Abstract in Czech
\begin{abstract-czech}
Tato bakalářská práce se zabývá vývojem bezdrátového senzoru pro měření vlhkosti půdy, který je užitečný pro efektivní hospodaření s vodními zdroji v zemědělství, zahradnictví a domácnostech. Výsledkem práce jsou dva hardwarové komponenty, univerzální modul s podporou LoRa rozhranní a senzor vlhkosti půdy, založený na kapacitním měření objemového obsahu vody. Modul, postavený na čipu STM32WLE, obsahuje integrovanou nevolatilní paměť pro záznam dat a aktualizace firmwaru, je kompaktní, má nízkou spotřebu a univerzální podporu I/O. Firmware modulu je implementován v programovacím jazyce Rust. Experimentem byla oveřena stabilní komunikace modulu na vzdálenosti přesahující jeden kilometr.
\end{abstract-czech}

% Abstract in English
\begin{abstract-english}
This thesis examines the development of a wireless sensor for soil moisture sensing, crucial for efficient water resource management in agriculture, horticulture, and domestic settings. Outcome of the work are two hardware components, a general purpose LoRa module and the soil moisture sensor, based on capacitive volumetric water content measurement. The module, built on the STM32WLE System on a Chip, features integrated non-volatile memory for data logging and firmware updates, compact size, low power consumption, and versatile I/O support, its firmware is implemented in the Rust language. Field tests confirmed the module's stable communication over distances exceeding one kilometer, demonstrating its effectiveness and potential for broader applications.
\end{abstract-english}

% Acknowledgements / Podekovani
\begin{thanks}

\end{thanks}

% Declaration / Prohlaseni
\begin{declaration}
Prohlašuji, že jsem předloženou práci vypracoval samostatně, a že jsem uvedl veškerou použitou literaturu.

I hereby declare that I have independently authored the submitted thesis and have cited all the sources used.

\vspace{2em}
\noindent \ctufield{day}.~\monthinlanguage{title}~\ctufield{year}, Prague
\end{declaration}


\begin{document}

\maketitle

\ctutemplate{specification.as.chapter}

\chapter{\label{chapter:introduction}Introduction}

With the upward trend in global temperatures, the world is witnessing an increased scarcity of water resources. This environmental shift poses a significant challenge to traditional farming practices, which are heavily reliant on predictable weather patterns and stable water supply. 

In the face of these challenges, the adoption of smart agriculture technologies, emerges as a vital strategy. These technologies offer the potential to transform agricultural practices by optimizing water usage, improving crop yields, and ensuring sustainable farming operations.

This project aims to develop a sensor network that is not only reliable and easy to maintain but also versatile enough to be applied across various contexts - from small-scale home gardens to medium-sized agricultural fields and regional deployments. While the system is specifically tailored for smaller agricultural endeavors, its design principles and technologies hold the potential for adaptation and scaling. This flexibility ensures that the solution can contribute valuable insights and efficiency improvements to a wide range of agricultural and domestic settings, even in the face of limited resources and space.


\chapter{\label{chapter:prior}Prior art}
%!TEX ROOT=main.tex

\section{Soil water content}
Soil is made up of a solid phase of minerals and organic matter, and the pours in-between the solids, which hold gasses and water \cite{paul_soil_2007}. The total amount of moisture is the sum of the moisture contained inside the solids (in intra-aggregate pore space) and the space around the solids (inter-aggregate pore space) \cite{myjove_corporation_determination_2024}. This work will not distinguish between the two for simplicity.

\subsection{Definition}
Soil water or moisture content is a ratio, which ranges from 0, meaning completely dry, to the value of material porosity at saturation \cite{webster_humidity_1998}. It expresses the quantity of water contained in the soil. We can measure it by mass (gravimetric method) or by volume , as depicted in Figure \ref{fig:soil-phase-diagram}.

We can express this mathematically for volumetric content (VWC) as
\begin{equation}
    \label{equation:volumetric-content} \theta = \dfrac{V_w}{V_s + V_w + V_a}
\end{equation}
where $V_w$ is the volume of water and $V_s + V_w + V_a$ is the total volume of the soil sample including contained air. Likewise, gravimetric water content (GWC) is defined as
\begin{equation}
    \label{equation:gravimetric-content} u = \dfrac{m_w}{m_s}
\end{equation}
where $m_w$ is the mass of the water and $m_s$ is mass of all solids in the sample \cite{edaphic_scientific_pty_ltd_how_2024}.

\begin{figure}
    \includegraphics[width=.5\textwidth]{fig/soil-phase-diagram.png}
    \caption{\label{fig:soil-phase-diagram} Soil composition by volume and mass \cite{noauthor_water_2023}}
\end{figure}

\subsection{Methods of measurement}
\subsubsection{Drying the soil}
Drying the soil sample in a drying oven is a direct method of measurement and is used as the reference method \cite{webster_humidity_1998}. By weighing the sample and also measuring its volume, then doing that again after drying, it is possible to very accurately measure both the volumetric and the gravimetric water content both at the same time.

The procedure involves gathering a known quantity of the sample, ranging from 30 grams to 5 kg, depending on the fines of the particles, heating it up and drying it at 65 to 110 degrees Celsius until the weight stops decreasing \cite{department_of_sustainable_natural_resources_soil_2024,myjove_corporation_determination_2024, paul_soil_2007}.

Since direct methods of measurement of soil moisture content are impractical for field use, we will focus on indirect methods next.

\subsubsection{Geophysical methods}
Geophysical methods exploit other properties of water contained in the soil to approximate the VWC, such as its conductivity, dielectric constant or interaction with neutrons. These methods are thus indirect and subject to inaccuracy if wrong assumptions are used \cite{webster_humidity_1998}. However, if applied correctly, these methods allow the continuous monitoring of the water content without human intervention.



\subsubsection{Satellite remote sensing method}
Thanks to recent and ongoing large-scale deployments of Synthetic Aperture Radar satellites, it is possible, that for global-scale soil water content estimation this method will become much more wide-spread. It also relies on the large contrast in dielectric properties of wet and dry soil.

mozna pridat nejakou mapku pro ilustraci?


\section{Capacitive soil moisture sensors}



\section{LoRa}
LoRa (Long Range) is a low-power wireless communication technology designed specifically for long-range communications with low power consumption. It is one of the leading technologies used in the rapidly expanding field of the Internet of Things (IoT), particularly in applications requiring devices to send small amounts of data over long distances while conserving battery life.

Its modulation technique is known as LoRa modulation and is derived from chirp spread spectrum (CSS) modulation. It operates in the sub-gigahertz radio frequency bands, and it is typically used in a star-of-stars network topology in conjunction with LoRaWAN (Long Range Wide Area Network), which defines the communication protocol and system architecture for the network.

LoRa is particularly noted for its excellent penetration in dense urban environments and indoor connectivity, compared to other technologies like Wi-Fi and cellular networks. This capability, combined with its low power requirements and long range, provides a distinct advantage in scenarios where alternative networking technologies might fail or be too costly.

\chapter{\label{chapter:architecture}System architecture}
%!TEX ROOT=main.tex

This chapter will walk the reader through the various options that were considered and final choices that were made during the concept and initial design stages of this project.

\section{Separation of concerns}
As was discussed in the Introduction \ref{chapter:introduction}, the main goal of this work is the soil moisture sensor hardware with the accompanying firmware and a proof-of-concept application.

With that being said, one can easily discern separate sub-tasks within this broader goal - mainly the fact, that the communication aspect can be separated from the sensor itself (see Figure \ref{fig:device-split}) a practice commonly seen in the industry.

\begin{figure}
    \includesvg[width=\textwidth]{fig/device-split.drawio.svg}
    \caption{\label{fig:device-split}Logical high level building blocks of most modern sensors}
\end{figure}

Separating these two concerns not only logically, but also on the hardware level, will bring many advantages - future sensor implementations can be made with only effort being put into the sensor itself; the wireless part and most of the testing and regulation overhead can be solved once, not repeated for every sensor type; once a sensor compatible with the interface exists, it can be made compatible with future versions of the interface - to list a few.

While this work is mainly concerned with the soil moisture sensing application, having a LoRa compatible unit, which is capable of OTA updates and has enough processing power to handle most sensing and simple control task, while being power-efficient enough to be battery powered, is an interesting sub-goal of this work.

The following sections will go through the process of finding requirements for such hardware and explain the compromises made.

\section{Module requirements}
\subsection{Over The Air update support}
There are two general approaches to implementing OTA updates. First is to split the firmware into a so-called Bootloader, a binary which runs before the main application and implements a way to communicate with the master node, such that it is able to overwrite the application. The application usually also needs to communicate with the master node in order to transmit sensor data and accept commands, which leads to duplication of all the driver and protocol logic required to use the communication interface.

The first option provides some level of separation between the application and the Bootloader, which could be useful in cases where the application is in some ways ``untrusted'', either due to security measures limiting the potential impact of bugs, or because it is actually provided by a different vendor and so on. This could be especially valid for cases, where the processor provides privileged execution modes etc.

But it also cuts the other way, since this approach hinders the upgradeability of the device itself, because now this relatively large Bootloader with all the error prone interface handling is locked in and unable to be updated remotely, unless, of course, the application also implements a way to update the Bootloader, but this defeats any potential security benefits, unless a sophisticated authentication method is employed in both the Bootloader and the Application, which leads to even more duplication.

This brings us to the second option, which also consists of a Bootloader and the Application, but here the Bootloader is much simpler, it only implements the bare minimum that is required to take a newer version of the application and overwrite the currently active one. However, if the Bootloader is unable to communicate with the master node, this means that the Application needs to download the new binary and store it somewhere locally on the end-node. 

This is usually a challenge for embedded devices, where the built-in FLASH memory, in which the application is stored and from which it is executed, is generally just about able to fit the Bootloader and the Application and rarely would be able to accommodate two copies of the binary.

Integrating additional memory increases the cost, but it can be used for other proposes besides OTA updates, such as storing configuration, calibration data and logs, or if the application integrates some sort of graphical display, may be required to store the assets, such as images and fonts.

Perhaps the most important benefit is that, with proper Bootloader the support, the application binaries can be swapped instead of the old one being deleted, which opens up the possibility of rolling back unsucsefull updates, which could be considered a crucial feature in some applications and is not practical to do in the first OTA approach that was presented.

\subsection{\label{section:application-case-studies}Application case-studies}
In order to find the optimal boundary between the sensor implementation part and the Interface and Processing part, as defined in Figure \ref{fig:device-split}, it is useful to look at the possible applications of the proposed module.

\subsubsection{Indoor environment sensor array}
Let us consider this basic, typical, use-case for such a communication module. This application can implement the following sensors
\begin{itemize}
    \item thermometer
    \item hygrometer (relative humidity sensor)
    \item human presence detector
    \item air quality sensor (CO$^2$e, smoke, ...)
    \item light sensor
\end{itemize}
we can omit some of the listed sensors in the actual application, but the module should be able to support the full configuration without any kind of co-processor. Main limiting factor will probably be the number of communication peripherals a General Purpose Input Output pins.

Thermometer is usually an integral part of any hygrometer measuring relative humidity \cite{webster_humidity_1998}, that is also suitable for this application. These sensors are frequently found in fully integrated solutions with a digital interface of some sort, usually I2C \cite{bosch_sensortec_gmbh_bst-bme280-ds002pdf_2024}. The same is true for any modern light sensor, which will also be able to measure intensities of different wavelengths of light \cite{stmicroelectronics_ambient_2024}, \cite{texas_instruments_inc_light_2024}. Thus more than half of the sensors listed only require a single I2C port to control them comfortably.

Traditionally, PIR sensors are used to detect motion, thus presence of humans in the vicinity of the sensor, but this might not work reliably for indoor applications. Thus, nowadays, the use of radar-based systems \cite{infineon_technologies_presence_2024} or IR ranging sensors \cite{stmicroelectronics_human_2024} are a lot more prevalent for human presence detectors. Such a sensor might expose a digital interface, such as I2C or SPI, or simply output an analog signal, which can be sampled using an ADC.

Other environmental sensors, such as air quality sensors, also implement similar interfaces - I2C or SPI or an analog output. Notably these sensors usually exhibit relatively high power draw ($>100~\mathrm{mW}$) and slowest startup times of all the other sensors of this application (orders of 10s of seconds to minutes) \cite{amphenol_inc_mics-vz-89te_2024}, so they are not suitable for battery-powered applications.

On the note of power draw, this application may wish to be battery powered or remain mains powered, this will affect the capabilities and the end use-case. 

When running on battery, the active on-time is limited to periodic sampling of the environment a handful times per hour. Being able to power down all sensors can prove useful in this application to greatly improve the battery life. On the other hand, if the application aims at fast reaction times, switching on the lights when presence detected for example, and the inclusion of all the sensors listed, it will need to be mains powered to be practical.

\subsubsection{Light dimmer}
For this application only a timer peripheral capable of generating PWM of sufficient frequency and resolution on a handful of channels is needed. Such peripheral exists on most modern microcontrollers.

If local control is also required, a rotary encoder for example, which can be sampled using a digital input interrupt or a dedicated peripheral designed to handle encoders.

\subsubsection{Soil moisture sensor}
The defining features of this application are outdoor use, battery power with the possibility of including a solar panel for zero-maintenance operation, long range and low dynamic duty-cycle.

\subsubsection{Gateway}
The module should be versatile enough to be also able to act as a communication interface for a host computer to connect to and manage the network of sensors, though other more specialized hardware could also be used for this use-case.

\subsection{\label{section:final-requirements}Final requirements}
\begin{itemize}
    \item 2V8-3V3 nominal voltage range - the lower the minimum threshold, the better (for being able to harvest as much energy as possible from ie. a coin-cell battery)
    \item low power design - support for switchable power rails for standby modes, low duty cycle operation, low power standby of the module itself
    \item target the 865-923 MHz (EU868, US915, IN865, ...) frequency range
    \item wide temperature range for outdoor applications
    \item support for wide range of use-cases - minimize the amount of specialized hardware on the module, leave that up the implementation
    \item minimal footprint
    \item support for OTA updates - large enough internal storage
    \item integrated RF - ideally a built-in antenna or some means to connect one
    \item host communication interface
    \item low cost
\end{itemize}

\subsubsection{Existing hardware satisfying these requirements}

SeedStudio Wio-E5-LE \cite{stmicroelectronics_lora_2024, seeedstudio_wio-e5-wireless_2024} is a cost effective LoRa module integrating the STM32WLE5JC SOC.

%\begin{table}[!h]
%\begin{center}
%\caption{\label{table:existing-modules} Existing modules satisfying project requirements}
%    \begin{tabular}{|l|c|c|c|} 
%    \hline
%    Name & Col2 & Col2 & Col3 \\
%    \hline
%     & 6 & 87837 & 787 \\ 
%    \hline
%    \end{tabular}
%\end{center}
%\end{table}

\section{\label{section:module-architecture}Module architecture and parts selection}

Solderable PCB modules are a standard way of integrating existing solutions into custom ones. Modules providing wireless connectivity in particular are very common, see Figure \ref{fig:wireless-modules}.

\begin{figure}
    \centering
    \subfloat[Omega 2S \cite{onion_corporation_omega2s_nodate}]{\includegraphics[width=.3\textwidth]{img/Omega2S.jpg}}
    \subfloat[\label{fig:RN4871} RN4871 \cite{microchip_technology_inc_rn4871_nodate}]{\includegraphics[width=.3\textwidth]{img/rn4871.jpg}}
    \subfloat[MAX F10s \cite{u-blox_max-f10s_2024}]{\includegraphics[width=.3\textwidth]{img/max-f10s.png}}
    \caption{\label{fig:wireless-modules} Common wireless modules}
\end{figure}

This approach allows us to separate the usually complex and more expensive multi-layer board layouts, required by modern SOCs, along with their power delivery and any other supporting circuitry, from the less complex end-application consisting of local power regulation, battery management, connectors and other mechanical features.

In order to satisfy the requirements \ref{section:final-requirements}, the module should provide means for analog and digital signal acquisition, digital communication interfaces and sufficient internal storage along with implementing the wireless connectivity.

\subsection{\label{section:mcu}Microcontroller}
Given the requirements for minimal footprint a fully integrated SOC solution is preferred to a configuration of separate MCU and an RF solution. STM32WL series offers such an SOC, which also satisfies the requirement of low power consumption by being based on the STM32L4, a well known ultra low power family of micro-controllers.

The manufacturer also offers a development board, the NUCLEO-WL55JC (\ref{fig:nucleo}), and a plethora of reference designs, the STDES-WL5xxxxx series, where the STDES-WL5U4ILH (\ref{fig:reference-design}) overlaps well with our requirements.

\begin{figure}
    \centering
    \subfloat[\label{fig:nucleo} NUCLEO-WL55JC \cite{stmicroelectronics_nucleo-wl55jc_2024}]{\includegraphics[width=.3\textwidth]{img/nucleo-wl55jc.jpg}}
    \subfloat[\label{fig:reference-design} STDES-WL5U4ILH \cite{stmicroelectronics_stdes-wl5u4ilh_2024}]{\includegraphics[width=.5\textwidth]{img/STDES-WL5U4ILH.jpg}}
    \caption{\label{fig:nucleo-and-reference} Nucleo development kit and the reference design board}
\end{figure}

Each of the designs focuses on optimizing different parameters depending on the application priorities and their codification follows table \ref{fig:reference-design-codification}, from which the defining features are apparent. For this work, application footprint was one of the top priorities, so the lower-power $15~\mathrm{dB}$ version with IPD was selected.

IPD stands for Integrated Passive Device, it consists of a balun and a harmonic filter. This circuitry is usually realized using passive components, which is cheaper, but takes up more board space (see Figure \ref{fig:frontend-comparison} for comparison), is more prone to design mistakes and tuning mismatch. This particular IPD was specifically designed for STM32WL line of microcontrollers in this configuration \cite{stmicroelectronics_balfhb-wl-05d3_2024}.

\begin{figure}
    \includegraphics[width=\textwidth]{fig/STDES-xxxxxxx.png}
    \caption{\label{fig:reference-design-codification} STM32WL5x and STM32WLEx reference designs codification}
\end{figure}

\begin{figure}
    \centering
    \subfloat[discrete /wo RF switch (variant D)]{\includegraphics[angle=-90,width=.28\textwidth]{img/frontend-dlb.png}}\hfill
    \subfloat[discrete /w RF switch (variant S)]{\includegraphics[angle=-90,width=.25\textwidth]{img/frontend-sbb.png}}\hfill
    \subfloat[IPD (variant I)]{\includegraphics[angle=-90,width=.28\textwidth]{img/frontend-ilh.png}}
    \caption{\label{fig:frontend-comparison} Frontend layout and part selection comparison, components on the RF path highlighted. Reference designs selected for comparison were picked based on the similarity with the STDES-WL5U4ILH design.}
\end{figure}

One downside of using an IPD in this case is that it requires the use of an RF switch, because ST's IPDs are designed to work with separate receive and transmit paths. Again this is a tradeoff between complexity, cost and board space required. Fortunately only a Single Pole Double Throw switch is required, thus a simple switch such as the BGS12SN6 from Infineon \cite{infineon_technologies_bgs12sn6_2024} is sufficient.

A 4 layer reference design was selected because it is anticipated it will allow a higher density layout, shrinking the module dimensions even further. UFQFPN package is preferred over UFBGA to stay compatible with lower cost manufacturing solutions. It is expected the design will use most of the available pins on the package, a complex fanout would be required if we were to go with the BGA variant, perhaps some space savings could be had at the cost of losing access to most of the pins, making any modifications to the design difficult to impossible, should the need arise.

In conclusion, the MCU choice is a culmination of tradeoffs, where the prevention of mistakes, efficiency and familiarity were more important than the absolute performance and cost. This should not in any way hamper further improvements in succeeding versions of this hardware, while allowing the completion of Proof-of-Concept stages of this project. 

\subsection{Power}
The selected MCU supports wide input voltage of $2 \mathrm{--} 3.6~\mathrm{V}$ thanks to its internal voltage regulation. It supports two modes - LDO, which does not require any additional external component at the cost of lower efficiency and the SMPS mode, utilizing a synchronous buck regulator, which is more efficient - this application will therefor attempt to implement the latter.

A separate, switchable power-rail is a feature deemed necessary by some considered applications in \ref{section:application-case-studies}. This can be achieved using a built-in MOSFET - such power rail could be used to save power even by powering off some parts of the module itself.

\subsection{Non-volatile memory}


\subsection{\label{section:antenna}Antenna}
In the initial module requirements \ref{section:final-requirements} it was deemed preferable to integrate the antenna onto the module itself. This should not, however, compromise on the usability and performance of the module. 

In general, there are two approaches for integrated antennae - a trace antenna constructed using the PCB directly, or an antenna in the form of a solderable component, such as an ceramic chip antenna. If the board space is limited, the performance is insufficient or the use of an integrated antenna is prohibited by any other limitation, the application must resort to an external antenna connected to the RF circuitry through a coaxial cable. The table \ref{table:antenna-solutions} provides a good summary.

\begin{table}[H]
\begin{center}
\caption{\label{table:antenna-solutions}Advantages and Disadvantages of different antenna solutions \cite{andersen_selecting_2008}}
    \begin{tabular}{|l|l|l|}
    \hline
    \textbf{Antenna types} & \textbf{Advantages} & \textbf{Disadvantages} \\
    \hline
    PCB antenna  & Low cost, No assembly,        & Difficult to design small \\
                 & Good performance achievable,  & and efficient antenna, \\
                 & Small size at high frequency  & Large size at low frequency \\
    \hline
    Chip antenna & Small size,                   & Medium performance, \\
                 & Off-the-shelf solution        & Medium cost \\
    \hline
    Whip antenna & Good performance,             & High cost, \\
                 & Off-the-shelf solution        & Large size \\
    \hline
    \end{tabular}
\end{center}
\end{table}

The following sections elaborate on each antenna type in more detail in the context of this particular application. 

At this point in the design process, the module proportions, derived from the individual component footprints, are not expected to surpass $500~\mathrm{mm^2}$. It is also expected the design will need to conform to the PCB parameters specified in Table \ref{table:pcb-parameters}. Since the first prototypes will only operate in Europe, the aim is mainly the EU868 band.

\begin{table}[H]
\begin{center}
\caption{\label{table:pcb-parameters}Approximate PCBWay manufacturer stackup parameters \cite{pcbway_stackup_2024}}
    \begin{tabular}{|l|l|} \hline
    \textbf{Parameter}            & \textbf{Value} \\ \hline
    Stackup thickness             & $1.6~\mathrm{mm}$ \\ \hline
    Substrate dielectric constant & $4.5~\mathrm{[-]}$ \\ \hline
    Substrate loss tangent        & $0.02~\mathrm{[-]}$ \\ \hline
    \end{tabular}
\end{center}
\end{table}

\subsubsection{Printed Circuit Board antenna}
Due to the limited board space, this option is expected to not be feasible. Nevertheless, some types of PCB antenna solutions were calculated to support this intuition.

A microstrip patch antenna is one of the simplest types of PCB antenna designs \cite{zachariah_peterson_microstrip_2022,wallace_an058_nodate}. This antenna consists of the patch itself and its feed-line with optional inset for impedance matching, see \ref{fig:patch-antenna}. 

The approximate parameters presented in Table \ref{table:antenna-pcb-calculations} were obtained using \cite{zachariah_peterson_microstrip_2022}, the input impedance could be tuned using the inset and a matching circuit, but more importantly, this simple antenna design is not practical because of its size, which already has roughly 15 times the surface area of the module components.

Another, more space-efficient option, is to use the inverted-F design, see Figure \ref{fig:inverted-f-antenna} for its simplest form. We can calculate its length $L$ given the target frequency $f$
\begin{equation}
    % 299,792,458/(4*868,000,000)
    L = \dfrac{\lambda}{4} = \dfrac{c_0}{4f} = \dfrac{c_0}{4 \cdot 868 \cdot 10^6} \approx 86.3~\mathrm{mm}
\end{equation}
which is a great improvement compared to the patch antenna.

This design could perhaps be optimized such that it would fit the size constraints, but such work might be enough to write another thesis focused on just this detail. Dimensions based on the layout in Figure \ref{fig:inverted-f-antenna} were included in Table \ref{table:antenna-pcb-calculations} for comparison sake.

\begin{figure}
    \centering
    \subfloat[\label{fig:patch-antenna}Microstrip patch antenna dimensions including the feed-line \cite{zachariah_peterson_microstrip_2022}]{\includegraphics[width=.53\textwidth]{fig/patch-antenna.png}}\hfill
    \subfloat[\label{fig:inverted-f-antenna}Basic inverted-F antenna design \cite{peterson_inverted-f_2023}]{\includegraphics[width=.43\textwidth]{fig/inverted-f-antenna.png}}
    \caption{Simple Printed Circuit Board antenna designs}
\end{figure}

\begin{table}[H]
\begin{center}
\caption{\label{table:antenna-pcb-calculations}Approximate microstrip patch antenna parameters and dimensions (as described in \ref{fig:patch-antenna}) and a notion of the dimensions of an inverted-F antenna (corresponding to \ref{fig:inverted-f-antenna}), not including the ground-plane and matching network}
    \begin{tabular}{|l|l|l|} \hline
    \textbf{Parameter}  & \textbf{Patch antenna}    & \textbf{Inverted-F antenna} \\ \hline
    Input impedance     & $306~\mathrm{\Omega}$     & not calculated \\ \hline
    Bandwidth           & $3.92~\mathrm{MHz}$       & not calculated \\ \hline
    Width (W)           & $104~\mathrm{mm}$         & $\approx 70~\mathrm{mm}$ \\ \hline
    Length (L)          & $80~\mathrm{mm}$          & $\approx 18~\mathrm{mm}$ \\ \hline
    Minimum footprint   & $8320~\mathrm{mm^2}$      & $1260~\mathrm{mm^2}$ \\ \hline
    \end{tabular}
\end{center}
\end{table}

These preliminary calculations led to the conclusion, that a PCB antenna at this frequency with these size constraints is not viable in this case. These calculations are not even beginning to factor in all the other components required, not to mention the even larger ground-plane which such antenna would require. 

This could be made possible with significant design effort and expertise, or by licensing an existing design, but the overall goal of creating this hardware would need to shift as a result.

\subsubsection{Chip antenna}
As outlined in \ref{table:antenna-solutions}, the chip antenna is another logical step in the search of the right solution. It is a common choice seen in commercially available hardware, especially in cases, where the space is at a premium - as can be seen on the RN4871 \ref{fig:RN4871} Bluetooth module.

There is a large selection of these antennae on the market, mainly targeted at the very common bands, such as $2.4~\mathrm{GHz}$ and $5~\mathrm{GHz}$, but solutions for the $868~\mathrm{MHz}$ frequency range are also not hard to come by \cite{digikey_rf_2024,mouser_europe_868_2024}.

Omnidirectional, size, groundplane, cost

50 ohm, linear

%https://cz.mouser.com/datasheet/2/418/9/ENG_DS_ant_868_hexx_A-3237873.pdf
%https://datasheets.kyocera-avx.com/ethertronics/AVX-E_M620720.pdf
%https://cdn.taoglas.com/datasheets/ILA.08.pdf

\begin{table}[H]
\begin{center}
\caption{\label{table:antenna-chip}Comparison of the three selected chip antennae}
    \begin{tabular}{|l|l|l|l|} \hline
    \textbf{Parameter} & \textbf{Linx ANT-868} & \textbf{Kyocera M620720} & \textbf{Taoglas ILA.08} \\ \hline
    Construction & Wire wound & Ceramic chip & Ceramic chip \\ \hline
    Rated frequency $\mathrm{[MHz]}$& $862 \mathrm{--} 876$ & $863 \mathrm{--} 870$ & $863 \mathrm{--} 870$ \\ \hline
    Peak gain $\mathrm{[dB]}$       & $5.6$     & $0.3$     & $-0.5$ \\ \hline
    Average gain $\mathrm{[dB]}$    & $-1.1$    & $-5.0$    & $-4.0$ \\ \hline
    Maximum VSWR                    & $2.2$     & $1.7$     & $1.9$ \\ \hline
    Dimensions $\mathrm{[mm]}$      & $25.4 \times 15.3 \times 8.9$ & $6.0 \times 2.0 \times 1.0$ & $5.0 \times 3.0 \times 0.5$ \\ \hline
    Ground plane $\mathrm{[mm]}$    & $84.0 \times 38.0$ & $100.0 \times 40.0$ & $80.0 \times 40.0$ \\ \hline
    \end{tabular}
\end{center}
\end{table}

\subsection{Conclusion}
During the component selection phase, the requirements set in \ref{section:final-requirements} were continually assessed and successfully met. One major compromise was made with the antenna selection, where the design needed to resort to an externally connected antenna dut to the size and application constraints. This choice however should ensure that the potential for problems arising during prototype stages should be limited.

\FloatBarrier
\section{Sensor architecture}
Given that the primary point of this work is to create a proof-of-concept sensing solution, which utilizes the module envisioned in \ref{section:module-architecture}, the sensor should be suitable for small-scale testing and prototype work, it should also be simple to manufacture and maximize the potential for success in the first iteration, due to the limited time allocated for the whole project.

A single board construction makes sense here, as it is simple, self-contained and can be made through standard PCB manufacturers. It is also commonly seen in hobby grade and some commercial soil moisture sensors.

\subsection{Expected capacitance range}

\subsection{Measuring method}

\subsection{Battery and charging}

\chapter{\label{chapter:results}Results}
%!TEX ROOT=main.tex

\section{Module design}
As discussed in section \ref{section:module-architecture}, where the LoRa module was architected and its main parts were selected, the schematic and board layout were created. These outputs are included in image form as Appendix \ref{chapter:module01-files} along with more details and available at (https://github.com/manakjiri/lora-module-hw/releases/tag/v0.1)?.

The Open source KiCad Electronic Design Automation software was used throughout the project to create these designs. 

\subsection{Schematic}
While the \ref{section:module-architecture} focused on the fundamental parts of the design, many details were left up to the later development stages, once the overall system implementation is more clear. This section will focus and expand on these parts of the design.

Of note is the selection of the main clock source for the MCU \ref{section:mcu}. In this case the only two options are to use an XO on an TCXO. rozsirit?

All other aspects of the MCU integration were executed according to the application notes (cite), the datasheet and closely followed the implementations of the reference design and the nucleo development kit. This includes the selection of decoupling capacitors, the SMPS circuitry and the reset handling.

Another consideration was the implementation of the switchable power rail VDD\_SW. This rail taps off the main power rail of the rest of the module, any disruption could cause a glitch or trigger the BOR protection circuitry. It is thus necessary to limit the inrush current caused by this rail's switch-on and subsequent charging of any local capacitances.

dopsat vypocty casu

\subsection{Board layout}
A 4 layer board stackup was proposed in Section \ref{section:mcu}, which indeed ended up being used in this design. Purpose of each layer is described in table \ref{table:board-layers} and clearly observable in the final renders \ref{board:v0.1}.

\begin{table}[H]
\begin{center}
\caption{\label{table:board-layers}Module board layer signal and power assignments}
    \begin{tabular}{|l|l|} \hline
    \textbf{Layer name}     & \textbf{Primary purpose} \\ \hline
    F. Front layer          & Components and local connections \\ \hline
    1. First inner layer    & Ground \\ \hline
    2. Second inner layer   & Power \\ \hline
    B. Back layer           & Signal and markings \\ \hline
    \end{tabular}
\end{center}
\end{table}

The board is only populated on the front side \ref{board:v0.1-components}, to allow its use as a solder-able module. Still, the final dimensions of the module are $20.32 \times 22.48~\mathrm{mm}$, which is better than the initial optimistic estimate given in \ref{section:antenna}.

To stay compatible with low-cost manufacturing options, conservative parameters were picked when it comes to minimal clearance, trace width and drill size. No density-increasing technologies, such as blind or buried vias, via-in-pad, micro-via, etc. were not employed either. 

These parameters are summarized in the following Table \ref{table:board-limits} and were enforced by the Design Rule Checker throughout the project.

\begin{table}[H]
\begin{center}
\caption{\label{table:board-limits}Board layout physical limits}
    \begin{tabular}{|l|l|} \hline
    \textbf{Parameter}          & \textbf{Dimension} \\ \hline
    Minimum trace clearance & $0.150~\mathrm{mm}$ \\ \hline
    Minimum trace width & $0.150~\mathrm{mm}$ \\ \hline
    Minimum via width & $0.300~\mathrm{mm}$ \\ \hline
    Hole to trace clearance & $0.254~\mathrm{mm}$ \\ \hline
    Hole to hole clearance & $0.500~\mathrm{mm}$ \\ \hline
    Board edge to trace clearance & $0.150~\mathrm{mm}$ \\ \hline
    \end{tabular}
\end{center}
\end{table}

Being only 4 layers, the traces needed to be routed in densely and well-throughout manner, attempting to minimize the number of relatively large vias required. For this reason, the module's external connection signal assignments were decided only near the end of the design stage conforming mostly to the existing pin locations on the MCU itself.

The final pad assignments

\begin{table}[H]
\begin{center}
\caption{\label{table:module-pin-legend}Module pin legend including feature summary. Refer to each column following ``Pin'' (excluding ``Other'') as [Column header][Column-row contents], such as ``SPI1\_MOSI'' and so on. Some features were omitted for clarity, for complete list refer to the MCU manufacturer's documentation}
    \begin{tabular}{|l|l|l|l|l|l|l|l|l|} \hline
    \textbf{Name} & \textbf{Pin} & \textbf{TIM} & \textbf{ADC} & \textbf{I2C} & \textbf{SPI} & \textbf{UART} & \textbf{Other}\\ \hline
    1        & PA7  & 17\_1, 1\_1N    &     & 3\_SCL & 1\_MOSI &             & CMP2\_OUT          \\ \hline
    2        & PA6  & 16\_1          &     &       & 1\_MISO &             &                    \\ \hline
    3        & PA4  & L1,2\_OUT      &     &       &        &             & RTC\_OUT2           \\ \hline
    4        & PA2  & 2\_3           &     &       &        & 2\_TX & CMP2\_OUT           \\ \hline
    5        & PA1  & L3\_OUT, 2\_2   &     &       & 1\_SCK  &             &                    \\ \hline
    6        & PA0  & 2\_1           &     &       &        &             & WKUP1   \\ \hline
    7        & PB8  & 16\_1, 1\_2N    &     & 1\_SCL &        &             &                    \\ \hline
    8        & PB7  & L1\_IN2, 17\_1N &     & 1\_SDA &        & 1\_RX        &                    \\ \hline
    9        & PB6  &               &     & 1\_SCL &        & 1\_TX        &                    \\ \hline
    10       & PB5  & L1\_IN1        &     &       & 1\_MOSI &             & CMP2\_OUT          \\ \hline
    11       & PB4  &               & 3   & 3\_SDA & 1\_MISO &             & CMP1,2\_INP        \\ \hline
    12       & PA11 & 1\_4           & 7   & 2\_SDA & 1\_MISO &             & CMP1,2\_INM        \\ \hline
    SWO/13   & PB3  & 2\_CH2         & 2   &       & 1\_SCK  &             & WKUP3 \\ \hline
    SWDIO/14 & PA13 &               & 9   &       &        &             & IR\_OUT             \\ \hline
    SWCLK/15 & PA14 & L1\_OUT        & 10  &       &        &             &                    \\ \hline
    BOOT0/16 & PH3  &               &     &       &        &             &                    \\ \hline
    \end{tabular}
\end{center}
\end{table}

Final module specification

\begin{table}[H]
\begin{center}
\caption{\label{table:module-specification}Final module specification}
    \begin{tabular}{|l|l|} \hline
    Supply voltage range                    & $2.0\text{--}3.5~\mathrm{V}$ overit\\ \hline
    Maximum supply current (excluding EXT)  & $65~\mathrm{mA}$\\ \hline
    Standby supply current                  & $?~\mathrm{uA}$\\ \hline
    Operating temperature range             & $-40 \text{--} 85~\mathrm{^\circ C}$\\ \hline
    Output RF power                         & $15~\mathrm{dB}$\\ \hline
    Operating band                          & $868~\mathrm{MHz}$\\ \hline
    RF connector                            & U.FL \\ \hline
    Module connection type                  & Castellated hole (0.1 inch pitch) \\ \hline
    Supported interfaces                    & UART, SPI, I2C \\ \hline
    Programming interface                   & ARM Serial Wire Debug \\ \hline
    \end{tabular}
\end{center}
\end{table}
\chapter{Conclusions}
%!TEX ROOT=main.tex

This work explored the requirements of wireless sensors for the use in soil moisture sensing applications, useful for managing water resources in agriculture, horticulture and home environments.

A communications module based on the STM32WLE5JC System on a Chip supporting the LoRa wireless interface operating in the EU868 band was designed, manufactured and validated. Its biggest differentiator against other similar hardware is the integrated 1 Mbyte non-volatile memory, which can be used for data logging and configuration, but its main purpose is to facilitate safe and efficient Over The Air updates of the firmware running on the module. This module is versatile enough to be useful in other applications outside of this work with its small footprint of $20.32 \times 22.48~\mathrm{mm}$, low power consumption (9 mA receiving) and 16 Input/Output pins supporting interfaces such as UART, I2C and featuring 5 ADC channels.

For the firmware and host-side software implementation the Rust programming language was used. STM32 support and the async-await executor were provided by the Embassy project, together with the Bootloader, while the lora-rs library was used to integrate LoRa. The firmware was split into the 
\begin{itemize}
    \item generic module-runtime library code, which includes hardware initialization, the OTA implementation and other utilities,
    \item the module-gateway and module-node applications, which implement the gateway (communication with the host computer) and the node (moisture sensing application) respectively.
\end{itemize}

The soil moisture sensor is a piece of hardware containing the LoRa module, 4 capacitive sensing zones, 2 temperature sensors, measuring and charging circuitry, and is designed to be stuck into the soil to measure its moisture content. The active sensor area can measure the volumetric moisture level in 20, 50, 90 and 120 millimeters below surface, the temperature is measured near the surface and at 150 millimeters below surface. The sensor is able to charge its 330 mAh lithium cell from an integrated 300 mWp solar panel and stay operational throughout the day currently, however with more work it could function for weeks on a single charge, without much sunlight.

Range test was conducted in a typical deployment scenario, where the LoRa module proved to maintain stable connection at a distance of over 1 kilometer, with its antenna positioned only 190 millimeters above ground level. This was achieved at 15 dBm transmit power with the spreading factor set to 11, yielding average data rate of 300 bits per second (including protocol overhead and dead-time). It is even possible to far exceed this range in more favorable conditions, or increase the transfer speed significantly (up to 9 times) at the expanse of some range.

\printindex

\begin{appendices}
\chapter{\label{chapter:rust}Using the Rust programming language for embedded applications}

Rust is increasingly recognized in the field of embedded systems for its promise of memory safety and concurrent programming without the overhead of a runtime or garbage collector. Originally designed for systems programming, Rust offers deterministic performance and fine-grained control over hardware, akin to C, but with a stronger emphasis on safety and modern programming features.

As of recent years, the Rust language has seen growing adoption in embedded development, supported by a robust toolchain and a vibrant ecosystem. Rust's compiler, Cargo package manager, and integration with LLVM provide a seamless development experience, from writing high-level application logic to low-level hardware interfacing. The availability of crates, Rust's libraries, for various hardware abstractions and middleware, further contributes to Rust's suitability for embedded applications.

The traditional language of choice in this field has been C, thus it is customary to compare these two languages. Rust is a much younger language and as a result, it provides native support for many of the now ubiquitous features, such as error handling, package management, and generics, that programmers expect from any language, but the fundamentals are the most important aspects for embedded
\begin{itemize}
    \item \textbf{Memory Safety:} Rust's ownership model ensures memory safety at compile time, virtually eliminating common bugs such as buffer overflows and null pointer dereferences.
    \item \textbf{Concurrency:} Rust's approach to concurrency, based on the ownership and borrowing principles, allows developers to write inherently safe concurrent code without the typical risks of data races. While primarily developed for multi-threaded applications, this checking is also applicable to interrupts.
\end{itemize}

Additionally, it provides tools for code and library management, which traditionally needed to provided separately. The use of Cargo for dependency management and builds, along with Rust's built-in testing and documentation tools, modernizes the embedded development process.

However, being a new language is the source of its main objective drawbacks as well, including
\begin{itemize}
    \item \textbf{Steep Learning Curve:} Rust's strict compiler and its concepts of ownership and borrowing can be challenging for new users, particularly those familiar with the more forgiving nature of C.
    \item \textbf{Ecosystem Maturity:} While growing rapidly, the ecosystem for embedded development in Rust is less mature than C's, which has decades of accumulated libraries and tools.
\end{itemize}

\section{Frameworks in Rust for embedded systems}
Rust's ecosystem includes several frameworks designed to leverage its safety features and performance in embedded contexts. Notable among these are RTIC, which offers a real-time concurrency model, and Tock, an operating system for microcontrollers. These frameworks demonstrate Rust's capability to support both bare-metal applications and more complex operating system environments on embedded devices.

Embassy stands out as an advanced async/await executor tailored for embedded systems, facilitating the writing of asynchronous, non-blocking applications. It supports a variety of hardware platforms, among others including STM32 and NRF families, and peripherals, making it a versatile choice for developers looking to leverage modern Rust features in their embedded projects.

\begin{itemize}
    \item \textbf{Asynchronous Programming:} Embassy utilizes Rust's async/await syntax, simplifying the management of complex asynchronous operations and improving code clarity and maintainability.
    \item \textbf{Efficiency:} By enabling non-blocking programming models, Embassy helps optimize resource usage and power efficiency, which is crucial for battery-powered or energy-sensitive applications. All this is possible without the overhead of threads, since async/await is cooperative, much like many applications written in C for embedded.
    \item \textbf{Scalability:} The design of Embassy allows it to scale from small, single-core microcontrollers to more complex multi-core processors, supporting a wide range of application requirements.
\end{itemize}



\chapter{\label{chapter:module01-files}Module v0.1 design files}
%!TEX ROOT=main.tex

Following pages include the outputs from the design stage of the LoRa module.

\section{A word about the v0.2}

Due to the use of incompatible crystal oscillator in this version of the LoRa module a v0.2 was also designed. It includes the TCXO modification, as discussed in \ref{section:module-schematic}, an optimized pinout and some other minor edits. The v0.2 was not manufactured because the tuning of the front-end was not realized within the deadline of the project. 

The modifications needed for the v0.1 to function had little impact on the results and the additional resources that would need to be invested in manufacturing a second version with other possible defects, that could be resolved and tested on the v0.1, was not justified.

\begin{figure}
    \includegraphics[page=1,angle=-90,width=\textwidth]{boards/v0.1/lora-module.pdf}
    \caption{\label{schematic:v0.1-1}Top level schematic sheet.}
\end{figure}
\begin{figure}
    \includegraphics[page=2,angle=-90,width=\textwidth]{boards/v0.1/lora-module.pdf}
    \caption{\label{schematic:v0.1-2}STM32WLE5JC schematic sheet.}
\end{figure}
\begin{figure}
    \includegraphics[page=3,angle=-90,width=\textwidth]{boards/v0.1/lora-module.pdf}
    \caption{\label{schematic:v0.1-3}RF frontend schematic sheet.}
\end{figure}
\begin{figure}
    \includegraphics[page=4,angle=-90,width=\textwidth]{boards/v0.1/lora-module.pdf}
    \caption{\label{schematic:v0.1-4}Non-volatile memory schematic sheet.}
\end{figure}

\begin{figure}
    \subfloat[Front layer]{\includesvg[width=.48\textwidth]{boards/v0.1/lora-module-F_Cu.svg}}\hfill
    \subfloat[Back layer]{\includesvg[width=.48\textwidth]{boards/v0.1/lora-module-B_Cu.svg}}\hfill
    \subfloat[Inner GND layer]{\includesvg[width=.48\textwidth]{boards/v0.1/lora-module-In1_Cu.svg}}\hfill
    \subfloat[Inner VCC layer]{\includesvg[width=.48\textwidth]{boards/v0.1/lora-module-In2_Cu.svg}}
    \caption{\label{board:v0.1}Module v0.1 PCB layer design.}
\end{figure}

\begin{figure}
    \includesvg[width=\textwidth]{boards/v0.1/lora-module-F_Fab.svg}
    \caption{\label{board:v0.1-components}Module v0.1 component position reference.}
\end{figure}

\begin{table}
\begin{center}
\caption{\label{table:module-bom}LoRa module Bill Of Materials (BOM).}
\begin{tabular}{|l|l|l|} \hline
    \textbf{Qty} &	\textbf{Reference(s)} &	\textbf{Value} \\ \hline
    4   & C101, C203, C208, C214 & 4.7u \\ \hline
    4  & C102, C204, C206, C207 & 100n \\ \hline
    4  & C209, C210, C216, C218 & 100n \\ \hline
    5  & C220, C221, C307, C309, C401 & 100n \\ \hline
    2   & C201, C202 & DNP \\ \hline
    1   & C211 & 470n \\ \hline
    3   & C213, C215, C217 &	33p \\ \hline
    1   & C219	& 3.3p \\ \hline
    1   & C301	& GCM155R71E473KA55 \\ \hline
    1   & C302	& GCM1555C1H680JA16 \\ \hline
    1   & C303	& GRM0335C1H101GA01D \\ \hline
    2   & C304, C310	& GRM0335C1H390JA01D \\ \hline
    3   & C305, C311, C312	& DNP \\ \hline
    1   & D101	& B1861NB-05D000134U1930 \\ \hline
    1   & J101	& U.FL-R-SMT-1 \\ \hline
    1   & L201	& BK1608HS601-T \\ \hline
    1   & L202	& MLZ2012M150W \\ \hline
    1   & L301	& LQW15AN47NG00 \\ \hline
    1   & L302	& 0R \\ \hline
    1   & Q101	& PMN48XP \\ \hline
    1   & R101	& 1M \\ \hline
    1   & R102	& 100k \\ \hline
    3   & R103, R201, R202	& 10k \\ \hline
    2   & R301, R302	& 100R \\ \hline
    1   & U201	& STM32WLE5CCUx \\ \hline
    1   & U301	& BALFHB-WL-05D3 \\ \hline
    1   & U302	& BGS12SN6E6327XTSA1 \\ \hline
    1   & U401	& SST26VF080A \\ \hline
    1   & XTAL201 &	NX2016SA-32MHZ-EXS00A-CS11336 \\ \hline
\end{tabular}
\end{center}
\end{table}

\chapter{\label{chapter:sensor-files}Soil moisture sensor board design files}
%!TEX ROOT=main.tex

\begin{figure}[H]
    \subfloat[Front layer]{\includesvg[width=\textwidth]{boards/sensor/soil-sensor-F_Cu.svg}}\hfill
    \subfloat[Back layer]{\includesvg[width=\textwidth]{boards/sensor/soil-sensor-B_Cu.svg}}
    \caption{\label{board:sensor}Sensor PCB layer design}
\end{figure}

\begin{figure}[H]
    \subfloat[Electronics area]{\adjustbox{trim={0 0 15cm 0},clip,width=.6\textwidth}{\includesvg{boards/sensor/soil-sensor-F_Fab.svg}}}\hfil
    \subfloat[Tip area]{\adjustbox{trim={17.5cm 0 0 0},clip,width=.35\textwidth}{\includesvg{boards/sensor/soil-sensor-F_Fab.svg}}}
    \caption{\label{board:sensor-components}Sensor board component position reference}
\end{figure}

\begin{figure}
    \includegraphics[page=1,angle=-90,width=\textwidth]{boards/sensor/soil-sensor.pdf}
    \caption{\label{schematic:sensor-1}Top level schematic sheet}
\end{figure}
\chapter{\label{chapter:more-range-test}Additional data from the range test}
%!TEX ROOT=main.tex

\begin{figure}[H]
    \centering
    \subfloat[SF11]{\includesvg[width=\textwidth]{data/range/out/elevation-sf11.svg}} \hfil
    \subfloat[SF5]{\includesvg[width=\textwidth]{data/range/out/elevation-sf5.svg}}
    \caption{\label{fig:range-elevation}}
\end{figure}
\begin{figure}[H]
    \centering
    \subfloat[SF11]{\includesvg[width=\textwidth]{data/range/out/elevation-sf11-far1.svg}} \hfil
    \subfloat[SF11]{\includesvg[width=\textwidth]{data/range/out/elevation-sf11-far2.svg}}
    \caption{\label{fig:range-elevation-far}}
\end{figure}
\end{appendices}

\bibliographystyle{plainurl}
\bibliography{main.bib}

\end{document}