% arara: bibtex

\documentclass[hyphens]{beamer}

\usepackage{amsmath}
\usepackage{mathtools}
\usepackage{amsfonts}
\usepackage{amssymb}
\usepackage{caption}
\usepackage{subcaption}
\usepackage{svg}

\usetheme{default}

\title{Sensor Network for Smart Agriculture}
\author{Jiří Maňák}
\date{\today}

\setbeamertemplate{navigation symbols}{}
\setbeamertemplate{footline}[page number]{}
\addtobeamertemplate{footline}{
    ~~~~\small{\url{manakjiri.cz/thesis}}
}{}
\newcommand{\backupbegin}{
   \newcounter{framenumberappendix}
   \setcounter{framenumberappendix}{\value{framenumber}}
}
\newcommand{\backupend}{
   \addtocounter{framenumberappendix}{-\value{framenumber}}
   \addtocounter{framenumber}{\value{framenumberappendix}} 
}

\begin{document}
\setlength{\leftmargini}{0.5cm}
\setlength{\leftmarginii}{0.5cm}

\begin{frame}
\titlepage
\end{frame}


\begin{frame}{Goals}
\begin{columns}[T]
\begin{column}{.5\textwidth}
    1. Create a universal platform:\\
    LoRa Module and Firmware
    \begin{figure}
        \centering
        \includegraphics[width=\linewidth]{img/module-v0.1.jpg}
    \end{figure}
\end{column}
\hfil
\begin{column}{.5\textwidth}
    2. Demonstrate its capabilities:\\
    Wireless soil moisture sensor
    \begin{figure}
        \centering
        \includegraphics[width=.9\linewidth]{img/sensor-deploy-up.jpg}
    \end{figure}
\end{column}
\end{columns}
\end{frame}


\begin{frame}{Goals}
1. Low--power LoRa Module
\begin{itemize}
    \item A platform for rapid application development
\end{itemize}
\vspace{1em}
2. Custom soil moisture sensor network
\begin{itemize}
    \item Able to cover large--enough area (kilometers)
    \item Zero--maintenance
    \item No external dependency on commercial networks
    \item Potentially extensible with more sensor types
\end{itemize}
\end{frame}

\begin{frame}{Cover large--enough area with no external dependency}
\begin{columns}[T]
\begin{column}{.31\textwidth}
    \begin{itemize}
        \item LoRa
        \begin{itemize}
            \item Long range (kilometers)
            \item Low power (5~mA receive)
        \end{itemize}
        \item Custom protocol
        \begin{itemize}
            \item Low duty--cycle
            \item Not reliant on LoRaWAN
            \item Efficient
        \end{itemize}
    \end{itemize}
\end{column}
\hfil
\begin{column}{.69\textwidth}
    \begin{figure}
        \centering
        \includegraphics[width=\linewidth]{img/lora-signal.jpg}
        \caption*{LoRa signal - Frequency/Time, \cite{noauthor_what_nodate}}
    \end{figure}
\end{column}
\end{columns}
\end{frame}


\begin{frame}{Zero--maintenance and extensible}
\begin{columns}[T]
\begin{column}{.4\textwidth}
    \begin{itemize}
        \item Solar power
        \begin{itemize}
            \item No infrastructure required
            \item Self--sufficient
        \end{itemize}
        \item Over--the--Air updates
        \begin{itemize}
            \item New features and~improvements
            \item Long--term support
            \item Unusual for LoRa nodes
        \end{itemize}
    \end{itemize}
\end{column}
\hfil
\begin{column}{.6\textwidth}
    \begin{figure}
        \centering
        \includegraphics[width=\linewidth]{img/ota-popular.png}
        \caption*{Conventional $\times$ OTA update, \cite{mobility_connected_what_2024}}
    \end{figure}
\end{column}
\end{columns}
\end{frame}


\begin{frame}{Soil Moisture Sensor}
Absolute soil water content measurement is impractical
\begin{itemize}
    \item High power draw
    \item Bulky
    \item Expensive
\end{itemize}
\vspace{1em}
Properly calibrated relative measurement is sufficient for monitoring purposes\\
\vspace{1em}
$\to$ Capacitive soil moisture sensor
\end{frame}


\begin{frame}{Soil Moisture Sensor}
\centering
\begin{columns}
\begin{column}{.25\textwidth}
    $$C = \dfrac{\epsilon_r \epsilon_0 S}{d}~~\mathrm{[F]}$$
    $$\tau = RC~~\mathrm{[s]}$$
\end{column}
\hfil
\begin{column}{.75\textwidth}
    \begin{figure}
        \includegraphics[width=\linewidth]{../thesis/fig/dielectric-constant.jpg}
        \caption*{Dielectric constant $\epsilon_r$ of~materials found in~soil, \cite{meter_group_soil_2023}}
    \end{figure}
\end{column}
\end{columns}
\begin{figure}
    \includegraphics[width=.9\linewidth]{../thesis/fig/principle-cap-measure.png}
\end{figure}
\end{frame}


\begin{frame}{Soil Moisture Sensor}
\begin{itemize}
    \item PCB construction
    \item 4 capacitive zones (15 cm total depth)
    \item 330 mAh lithium cell, 150 mWp solar panel
\end{itemize}
\begin{figure}
    \centering
    \includesvg[width=\linewidth]{../thesis/boards/sensor/soil-sensor-F_Cu.svg}
\end{figure}
$$\underbrace{\text{\hspace{5em}}}_{\text{Sensor electronics}}~\underbrace{\text{\hspace{19em}}}_{\text{Sensor active area}}\text{\hspace{2.5em}}$$
\end{frame}


\begin{frame}{Soil Moisture Sensor}
\begin{columns}[T]
\begin{column}{.5\textwidth}
    \begin{figure}
        \centering
        \includegraphics[width=\linewidth]{../thesis/img/sensor-case.png}
    \end{figure}
\end{column}
\hfil
\begin{column}{.5\textwidth}
    \begin{figure}
        \centering
        \includegraphics[width=\linewidth]{../thesis/img/sensor-deploy-close.jpg}
    \end{figure}
\end{column}
\end{columns}
\end{frame}


\begin{frame}{LoRa Module}
\begin{figure}
    \centering
    \small
    \includesvg[width=.8\linewidth]{img/module-v0.1.drawio.svg}
\end{figure}
\begin{columns}[T]
\begin{column}{.5\textwidth}
    \begin{itemize}
        \item \textbf{STM32WLE5CC}
        \item 868 MHz, 15 dBm
        \item \textbf{20.32}$\mathbf{\times}$\textbf{22.48 mm}
    \end{itemize}
\end{column}
\hfill
\begin{column}{.5\textwidth}
    \begin{itemize}
        \item 1 MB FLASH
        \item 2.3--3.5 V
        \item 16 IO pins
    \end{itemize}
\end{column}
\end{columns}
\end{frame}


\begin{frame}{LoRa Module Range Test}
\begin{columns}[T]
\begin{column}{.3\textwidth}
    \centering
    \begin{figure}
        \includegraphics[height=7.5cm]{img/lora-pole.jpg}
    \end{figure}
\end{column}
\hfil
\begin{column}{.3\textwidth}
    \centering
    \begin{figure}
        \includegraphics[height=7cm]{../thesis/img/range-pole-bottom.jpg}
    \end{figure}
    \vspace{-1em}
    Node 2
\end{column}
\hfil
\begin{column}{.3\textwidth}
    \centering
    \begin{figure}
        \includegraphics[height=7cm]{../thesis/img/range-pole-top.jpg}
    \end{figure}
    \vspace{-1em}
    Node 3
\end{column}
\end{columns}
\end{frame}


\begin{frame}{LoRa Module Range Test - SF11}
\begin{figure}
    \includesvg[width=\linewidth]{../thesis/data/range/out/success-sf11.svg}
\end{figure}
\begin{figure}
    \includesvg[width=\linewidth]{../thesis/data/range/out/relief-sf11.svg}
\end{figure}
\Tiny{\textbf{Node 2 - 20 cm, Node 3 - 80 cm, Node 4 (Nucleo) - 50 cm}, 1.125 s round-trip, 300 bps, Page 41, Section 4.4.4}
\end{frame}

\begin{frame}{LoRa Module Range Test - SF5}
\begin{figure}
    \includesvg[width=\linewidth]{../thesis/data/range/out/success-sf5.svg}
\end{figure}
\begin{figure}
    \includesvg[width=\linewidth]{../thesis/data/range/out/relief-sf5.svg}
\end{figure}
\Tiny{\textbf{Node 2 - 20 cm, Node 3 - 80 cm, Node 4 (Nucleo) - 50 cm}, 0.125 s round-trip, 3 kbps, Page 41, Section 4.4.4}
\end{frame}


\begin{frame}{Soil Moisture Sensor Validation}
\begin{figure}
    \includesvg[width=\linewidth]{../thesis/data/deployment/out/moisture-cal.svg}
\end{figure}
\end{frame}


\begin{frame}{Live Demo}
\begin{figure}
    \centering
    \includegraphics[width=.5\linewidth]{img/qr.png}
\end{figure}
\begin{center}
    (or visit the link)
\end{center}
\end{frame}


\begin{frame}{Conclusion}
This thesis brought
\begin{itemize}
    \item \textbf{STM32 LoRa Module - a low--power platform for connected sensors}
    \begin{itemize}
        \item \url{module-runtime} Rust package - HAL, protocol, async executor; enabling rapid application development with emphasis on reliability
        \item Optimized for low duty--cycle operation
        \item Over--the--Air update capability with rollback
    \end{itemize}
    \item \textbf{Soil moisture sensor - an application of the module}
    \item Backend service combining soil moisture sensor data with weather forecast to optimize watering schedule
\end{itemize}
\vspace{1em}
All publicly available at \url{https://github.com/manakjiri}
\end{frame}


\begin{frame}[allowframebreaks]
    \frametitle{References}
    \bibliographystyle{IEEEtran}
    \bibliography{sources.bib}
\end{frame}

\appendix
\backupbegin

\begin{frame}{LoRa Module}
\begin{figure}
    \centering
    \includegraphics[width=\linewidth]{img/module-schema.png}
\end{figure}
\end{frame}


\begin{frame}{LoRa Module}
\begin{columns}[T]
\begin{column}{.3\textwidth}
    \centering
    \begin{figure}
        \includegraphics[width=.9\linewidth]{img/STDES-WL5U4ILH.jpg}
    \end{figure}
    STDES-WL5U4ILH
\end{column}
\hfil
\begin{column}{.3\textwidth}
    \centering
    \begin{figure}
        \includegraphics[width=\linewidth]{img/nucleo-wl55jc.jpg}
    \end{figure}
    Nucleo-WL55JC
\end{column}
\end{columns}
\end{frame}


\begin{frame}{Existing solution?}
\begin{columns}
\begin{column}{.4\textwidth}
    \centering
    \begin{figure}
        \includegraphics[width=\linewidth]{img/wio-e5.png}
    \end{figure}
    Seeed Studio Wio-E5
\end{column}
\hfil
\begin{column}{.1\textwidth}
    \centering
    \Large
    \vspace{-2em}
    $$>$$
    ?
    $$<$$
\end{column}
\hfil
\begin{column}{.4\textwidth}
    \centering
    \begin{figure}
        \includegraphics[width=\linewidth]{img/module-v0.1.jpg}
    \end{figure}
    My LoRa Module
\end{column}
\end{columns}
\end{frame}


\begin{frame}{Solar Power}
\begin{figure}
    \includesvg[width=\linewidth]{../thesis/data/deployment/out/power.svg}
\end{figure}
\end{frame}

\begin{frame}{Firmware}
\begin{figure}
    \centering
    \includesvg[width=\linewidth]{img/firmware.drawio.svg}
\end{figure}
\end{frame}


\begin{frame}{Over The Air Update}
\begin{figure}
    \centering
    \includegraphics[width=\linewidth]{img/ota.png}
\end{figure}
\begin{flushright}
    Page 36, Figure 4.9
\end{flushright}
\end{frame}


\begin{frame}{LoRa Module}
\begin{itemize}
    \item 2.8--3.3 V nominal voltage range,
    \item low power design - support for switchable power rails,
    \item target the EU868,
    \item wide temperature range
    \item minimize the amount of specialized hardware,
    \item support for OTA updates,
    \item integrated RF,
    \item host communication interface,
    \item minimal footprint,
    \item low cost.
\end{itemize}
\begin{flushright}
    Page 15, Section 3.2.3
\end{flushright}
\end{frame}

\backupend

\end{document}
